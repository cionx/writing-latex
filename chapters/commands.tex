\chapter{Good behavior inside math mode}





\section{Use the right symbols}





\subsection{General symbols}

Use the right symbols.
\Cref{wrong symbol list} shows some popular sources of this problem:
\begin{table}[tb]
  \begin{center}
  \begin{tabular}{@{}clclc@{}}
      \toprule
      \textbf{symbol}
      &
      \multicolumn{2}{c}{\textbf{right commands}}
      &
      \multicolumn{2}{c}{\textbf{wrong commands}}
    \\
    \midrule
      element relation
      &
      \commandtt{in}
      &
      $\in$
      &
      \commandtt{epsilon}
      &
      $\epsilon$
    \\
      {}
      &
      {}
      &
      {}
      &
      \commandtt{varepsilon}
      &
      $\varepsilon$
    \\
      {}
      &
      \commandname{ni}
      &
      $\ni$
      &
      \commandname{backepsilon}
      &
      $\backepsilon$
    \\
    \cmidrule(lr){2-3} \cmidrule(l){4-5}
      empty set
      &
      \commandtt{emptyset}
      &
      $\emptyset$
      &
      \commandtt{phi}
      &
      $\phi$
    \\
      {}
      &
      \commandname{varnothing}
      &
      $\varnothing$
      &
      {}
      &
      {}
    \\
    \cmidrule(lr){2-3} \cmidrule(l){4-5}
      set difference
      &
      \texttt{A {\tbs}setminus B}
      &
      $A \setminus B$
      &
      \texttt{A {\tbs}backslash B}
      &
      $A \backslash B$
    \\
      {}
      &
      \texttt{A {\tbs}smallsetminus B}
      &
      $A \smallsetminus B$
      &
      {}
      &
      {}
    \\
      {}
      &
      \texttt{A - B}
      &
      $A - B$
      &
      {}
      &
      {}
    \\
    \cmidrule(lr){2-3} \cmidrule(l){4-5}
      implication
      &
      \commandtt{implies}
      &
      $\implies$
      &
      \commandtt{Rightarrow}
      &
      $\Rightarrow$
    \\
      {}
      &
      {}
      &
      {}
      &
      \texttt{=>}
      &
      $=>$
    \\
      {}
      &
      \commandtt{impliedby}
      &
      $\impliedby$
      &
      \commandtt{Leftarrow}
      &
      $\Leftarrow$
    \\
      {}
      &
      {}
      &
      {}
      &
      \texttt{<=}
      &
      $<=$
    \\
    \cmidrule(lr){2-3} \cmidrule(l){4-5}
      equivalence
      &
      \commandtt{iff}
      &
      $\iff$
      &
      \commandtt{Leftrightarrow}
      &
      $\Leftrightarrow$
    \\
    \cmidrule(lr){2-3} \cmidrule(l){4-5}
      definition
      &
      \commandtt{coloneqq}
      &
      $\coloneqq$
      &
      \texttt{:=}
      &
      $:=$
    \\
      {}
      &
      \commandtt{eqqcolon}
      &
      $\eqqcolon$
      &
      \texttt{=:}
      &
      $=:$
    \\
    \cmidrule(lr){2-3} \cmidrule(l){4-5}
      norm
      &
      \texttt{{\tbs}| x {\tbs}|}
      &
      $\| x \|$
      &
      \texttt{|| x ||}
      &
      $|| x ||$
    \\
      {}
      &
      \texttt{{\tbs}lVert x {\tbs}rVert}
      &
      {}
      &
      {}
      &
      {}
    \\
    \cmidrule(lr){2-3} \cmidrule(l){4-5}
      pointy brackets
      &
      \texttt{{\tbs}langle x {\tbs}rangle}
      &
      $\langle x \rangle$
      &
      \texttt{< x >}
      &
      $< x >$
    \\
    \cmidrule(lr){2-3} \cmidrule(l){4-5}
      infinity
      &
      \commandtt{infty}
      &
      $\infty$
      &
      \texttt{oo}
      &
      $oo$
    \\
    \cmidrule(lr){2-3} \cmidrule(l){4-5}
      function colon
      &
      \texttt{f {\tbs}colon X {\tbs}to Y}
      &
      $f \colon X \to Y$
      &
      \texttt{f : X {\tbs}to Y}
      &
      $f : X \to Y$
    \\
    \bottomrule
  \end{tabular}
  \end{center}
  \caption{Right symbols and wrong symbols.}
  \label{wrong symbol list}
\end{table}
The commands \commandtt{rightarrow} and \commandtt{to} give the same arrow.
So use whichever is more appropriate in the given situation.



\subsection{Operations}

Many mathematical operations have both a binary version and an operator version where the operation can range over some index set.
One should not confuse the two of them.
\Cref{binary vs operator} shows some popular binary operations and their operator counterpart.
\begin{table}[tb]
  \begin{center}
    \begingroup
    \renewcommand{\arraystretch}{1.3}
    \begin{tabular}{@{}llclc@{}}
      \toprule
      \textbf{operation}
      &
      \multicolumn{2}{c}{\textbf{binary}}
      &
      \multicolumn{2}{c}{\textbf{generalized}}
      \\
      \cmidrule(lr){2-3}
      \cmidrule(l){4-5}
      sum
      &
      \inlinecode{+}
      &
      $a + b$
      &
      \commandname{sum}
      &
      $\sum_i x_i$
      \\
      multiplication
      &
      \commandname{cdot}
      &
      $a \cdot b$
      &
      \commandname{prod}
      &
      $\prod_i x_i$
      \\
      direct sum
      &
      \commandname{oplus}
      &
      $A \oplus B$
      &
      \commandname{bigoplus}
      &
      $\bigoplus_i X_i$
      \\
      tensor product
      &
      \commandname{otimes}
      &
      $A \otimes B$
      &
      \commandname{bigotimes}
      &
      $\bigotimes_i X_i$
      \\
      wedge
      &
      \commandname{wedge}
      &
      $a \wedge b$
      &
      \commandname{bigwedge}
      &
      $\bigwedge_i X_i$
      \\
      union
      &
      \commandname{cup}
      &
      $A \cup B$
      &
      \commandname{bigcup}
      &
      $\bigcup_i X_i$
      \\
      intersection
      &
      \commandname{cap}
      &
      $A \cap B$
      &
      \commandname{bigcap}
      &
      $\bigcap_i X_i$
      \\
      product
      &
      \commandname{times}
      &
      $A \times B$
      &
      \commandname{prod}
      &
      $\prod_i X_i$
      \\
      {}
      &
      \commandname{sqcap}
      &
      $A \sqcap B$
      &
      \commandname{bigsqcap}
      &
      $\bigsqcap_i X_i$
      \\
      coproduct
      &
      \commandname{amalg}
      &
      $A \amalg B$
      &
      \commandname{coprod}
      &
      $\coprod_i X_i$
      \\
      {}
      &
      \commandname{sqcup}
      &
      $A \sqcup B$
      &
      \commandname{bigsqcup}
      &
      $\bigsqcup_i X_i$
      \\
      \bottomrule
    \end{tabular}
    \endgroup
  \end{center}
  \caption{Binary operation and operator version.}
  \label{binary vs operator}
\end{table}
The command \commandname{bigsqcap} requires the package \packagename{stmaryrd}.



\subsection{Negations}

For some mathematical symbols there also exists a negated version, which is formed by adding a vertical line between the bottom left and top right.
A general way of introducing such a line is the command \commandname{not}.
\begin{showlatex}*{Using \commandname{not}}
  It follows that $A \not\ni x$.
\end{showlatex}
But in praxis this command should seldom be used, as it often produces very bad looking output.
Consider the following example:
\begin{showlatex}{Why not to use \commandname{not}}
  Hence $A \not\implies B$.
\end{showlatex}
There are two solutions to this problem:

Many symbols already have a predefined crossed-out version available.
A few of them are collected in~\cref{negation list}.
\begin{table}[tb]
  \begin{center}
  \begin{tabular}{@{}lclc@{}}
    \toprule
      \multicolumn{2}{c}{\textbf{right}}
    &
      \multicolumn{2}{c}{\textbf{wrong}}
    \\
    \cmidrule(r){1-2} \cmidrule(l){3-4}
      \commandname{notin}
      &
      $\notin$
      &
      \inlinecode{{\tbs}not{\tbs}in}
      &
      $\not\in$
    \\
      \commandname{nexists}
      &
      $\nexists$
      &
      \inlinecode{{\tbs}not{\tbs}exists}
      &
      $\not\exists$
    \\
      \commandname{neq}
      &
      $\neq$
      &
      \inlinecode{{\tbs}not =}
      &
      $\not =$
    \\
      \commandname{nleq}
      &
      $\nleq$
      &
      \inlinecode{{\tbs}not{\tbs}leq}
      &
      $\not\leq$
    \\
      \commandname{nrightarrow}
    &
      $\nrightarrow$
    &
      \inlinecode{{\tbs}not{\tbs}rightarrow}
    &
      $\not\rightarrow$
    \\
    \bottomrule
  \end{tabular}
  \end{center}
  \caption{Negated versions of popular symbols.}
  \label{negation list}
\end{table}
If the required symbol is not predefined then the command~\commandname{centernot} from the package~\packagename{centernot} often produces better looking output then~\commandname{not}:
\begin{showlatex}{Using~\commandname{centernot}}
  Consider
  \[
    A \centernot\ni x
    \quad\text{versus}\quad
    \quad
    A \not\ni x \,.
  \]
  Consider also
  \[
    A \centernot\implies B
    \quad\text{versus}\quad
    A \not\implies B \,.
  \]
\end{showlatex}





\section{Don’t use \texorpdfstring{\commandname{subset}}{{\tbs}subseteq}}

Some people use~$\subset$~(\commandname{subset}) to denote inclusion and~$\subsetneq$~(\commandname{subsetneq}) to denote proper inclusion, while some other people use~$\subseteq$~(\commandname{subseteq}) to denote inclusion and~$\subset$ to denote proper inclusion.
The second convention has the advantage of making sense and being consistent with the usual use of~$\leq$ and~$<$, whereas the first convention has the non-advantage of existing.
The problem is that both conventions are wide-spread but use the symbol~$\subset$ is different ways.

The conflict between the above two conventions has abused the symbol~$\subset$ to a point that it should simply \emph{never} be used.
Instead one shold always use~$\subseteq$ for inclusion, $\subsetneq$ for proper inclusion and~$\nsubseteq$~(\commandname{nsubseteq}) for non-inclusion.

Some people prefer the symbols~$\subseteqq$~(\commandname{subseteqq}) and~$\subsetneqq$~(\commandname{subsetneqq}) instead.
The author thinks that these symbols are unnecessary large and recommends not to use them.
(But they do at least leave poor old~$\subset$ alone.)





\section{Don’t underline}

Underlining mathematics works well on the blackboard, in handwritting and it was done in the age of typewriters.
Don’t do it in {\LaTeX}.





\section{Using and defining commands}

Use predefined mathematical operators:
\begin{showlatex}{Using math operators vs.\ not using them}
Use $\sin(x)$ instead of $sin x$, use $\dim V$ instead of $dim V$ and use $\lim_{n \to \infty} a_n$ instead of $lim_{n \to \infty} a_n$.
\end{showlatex}
New commands can be defined in various ways:


\subsection{\texttt{DeclareMathOperator}}

Commands of the form \commandtt{Word} that give the output $\mathrm{Word}$ can easily be defined using \commandtt{DeclareMathOperator}.
The command \commandtt{DeclareMathOperator} can only be used in the preamble.

To define the command \commandtt{Hom} we us the following text in the preamble:
\begin{showcode}{Declaring a math operator with~\commandname{DeclareMathOperator}}
% in the preamble:
\DeclareMathOperator{\End}{End}
\end{showcode}
The command \commandtt{Hom} can then be used in the usual way:
\begin{showlatex}{Using a declared math operator}
Thus $\End(V) = \End_k(V)$ becomes a vector space.
\end{showlatex}
When a command \commandtt{Word} is defined with \commandtt{DeclareMathOperator} then \hologo{LaTeX} automatically inserts some space around \commandtt{Word} when needed:
\begin{showlatex}*{Automatic spacing of~\commandname{DeclareMathOperator}}
\begin{align*}
&x \End V
\\
&x \End(V)
\\
&x \End {(V)}
\end{align*}
\end{showlatex}
Note that in the first expression \hologo{LaTeX} inserts some spacing both to the left and to the right of $\End$.
In the second expression \hologo{LaTeX} observes that the used math operator is follows by a parenthesis and thus inserts no additional spacing.
For the third expression we prevent \hologo{LaTeX} from making such an observation by using a pair of curly brackets.



\subsection{\commandtt{operatorname}}

The command \commandtt{operatorname} can be used to give the formatting of a mathemical operator without defining a new command.
\begin{showlatex}{Using \commandtt{operatorname}}
Thus $\operatorname{Hom}(V,W) = \operatorname{Hom}_k(V,W)$ becomes a vector space.
\end{showlatex}
If the same command is used multiple times then one should use \commandtt{DeclareMathOperator} instead of \commandtt{operatorname}, as \commandtt{Word} easier to write and read than \commandtt{operatorname{Word}}, and keeps the code clean.

This behavior leads to a bad looking output when \commandtt{DeclareMathoperator} is abused.
Suppose that we want a command \commandtt{Complex} that inserts the code \commandtt{mathbb\{C\}}.
\begin{showcode}{Wrong way of using \commandtt{DeclareMathOperator}}
% in the preamble:
\DeclareMathOperator{\Complex}{\mathbb{C}}
\end{showcode}
This will lead to the following problem:
\begin{showlatex}{Wrong output when \commandname{DeclareMathOperator} is abused}
The span of $x_1, \dotsc, x_n \in \Complex^m$ equals $\Complex x_1 + \dotsb + \Complex x_n$.
\end{showlatex}
We expect the output $\mathbb{C} x_1 + \dotsb + \mathbb{C} x_n$ but get some unwanted spacing instead.



\subsection{Don’t abuse \commandtt{mathrm}}

The commands \commandtt{mathrm} and \commandtt{operatorname} do not give the same formatting.
With \commandtt{operatorname} we get necessary spacing when not using parentheses, which does not happen when using \commandtt{mathrm}.
\begin{showlatex}{Missing spacing after~\commandname{mathrm}}
Compare $\operatorname{End} V$ to $\mathrm{End} V$.
\end{showlatex}



\subsection{\commandtt{newcommand}}

A very general way of defining new commands is given by \commandtt{newcommand}.
Its syntax is as follows:
\begin{showcode}{Syntax of \commandtt{newcommand}}
\newcommand{\name}[n]{ definition including #1, ..., #n }
\end{showcode}
Here \texttt{n} is the number of arguments that the defined \commandtt{name} will take.
The arguments can be accessed as \texttt{\#1}, \texttt{\#2}, etc., where \texttt{\#i} is the $i$-th argument.
Consider the following example:
\begin{showlatex}{Using \commandtt{newcommand}}
\newcommand{\bimodule}[2]{#1-#2-bimodule}
Let $M$ be an \bimodule{$A$}{$B$}.
\end{showlatex}
One may think about \commandtt{DeclareMathOperator} as a combination of \commandtt{newcommand} and \commandtt{operatorname}:
\begin{showlatex}{\commandname{DeclareMathOperator} = \commandtt{newcommand} plus \commandtt{operatorname}}
\newcommand{\Ouv}{\operatorname{Ouv}}
$\Ouv X$
\end{showlatex}

Trying to define an already existing command with \commandtt{newcommand} will lead to an error.
To overwrite an already existing command one can use \commandtt{renewcommand} instead.
But this shouldn’t really be used (unles you really, \emph{really} know what you’re doing):
Even if you don’t like a particular command there is a chance that some package that you’re using relies on it.
So overwriting a command can easily surprise you with some new problems.



\subsection{\commandtt{DeclarePairedDelimiter}}

One special kind of commands are things like \commandtt{abs\{ \}} for absolute values, which are supposed to put a certain kind of delimiter to the sides of the given argument.
One can use \commandtt{DeclarePaireDelimiter} to define such commands:
\begin{showcode}{Using \commandtt{DeclarePairedDelimiter}}
\DeclarePairedDelimiter{\abs}{\lvert}{\rvert}
\end{showcode}
The defined command can then be used as \commandtt{abs\{ \}}:
\begin{showlatex}{Using a declared delimiter}
  \[
    \abs{-5} = 5
  \]
\end{showlatex}

But \commandtt{DeclarePairedDelimiter} does not only define \commandtt{abs} but also a starred version \commandtt{abs*} that scales the surrounding delimiters according to its content.
One can also specify a scaling size, like \commandtt{big}, \commandtt{bigg}, etc. to scale the delimiters:
\begin{showlatex}{Scaling of declaired delimiters}
\begin{align*}
  \abs{-\frac{1}{2}}
  &=
  \frac{1}{2} \,,
  \\
  \abs*{-\frac{1}{2}}
  &=
  \frac{1}{2} \,,
  \\
  \abs[\bigg]{-\frac{1}{2}}
  &=
  \frac{1}{2}
\end{align*}
\end{showlatex}

Sometimes one needs a little more control then the basic \commandtt{DeclairePairedDelimiter} offers.
The command \commandtt{DeclarePairedDelimiterX} allows to specify a number of arguments \texttt{[n]} and needs to know how to built up the expression between the delimiters using these specified arguments:
\begin{showcode}{Using \commandtt{DeclarePairedDelimiterX} for more advanced delimiters}
\DeclarePairedDelimiterX{\inner}[2]{\langle}{\rangle}{#1 \,\delimsize\vert\, #2}
\end{showcode}
The command \commandtt{delimsize} ensures that the vertical line \commandtt{vert} will be properly scaled when the starred version of the command will be used:
\begin{showlatex}{Using more advanced delimiters}
\[
  \inner{\psi_1}{\psi_2}
  \quad
  \inner*{\frac{f}{g}}{\frac{h}{k}}
\]
\end{showlatex}

There is also the even more advanced command \commandtt{DeclarePairedDelimiterXPP}, which also allows to specify code that is to be inserted before the left delimiter and after the right delimiter:
\begin{showcode}{Syntax of \commandtt{DeclarePairedDelimiterXPP}}
\DeclarePairedDelimiterXPP{\name}[n]{left code}{left delimiter}{right delimiter}{right code}{inner code}
\end{showcode}
Suppose for example that we have defined a command \commandtt{norm} as follows:
\begin{showcode}{Defining \commandtt{norm} in the naive way}
\DeclarePairedDelimiter{\norm}{\lVert}{\rVert}
\end{showcode}
This will lead to the following output:
\begin{showlatex}{Using the naive \commandtt{norm}}
\[
  \norm{f}
  =
  \sup_{x \in I} \norm{f(x)}
\]
\end{showlatex}
Note that there seems to be some space missing between the expressions $\sup$ and $\norm{f(x)}$.
This may seem surprising at first but makes sense:
In the similar expression $\sup(M)$ we dont expect any space between \commandtt{sup} and the following left delimiter \texttt{(}, so here we also don’t get any spacing between \commandtt{sup} and the following left delimiter \commandtt{lVert}.
We can circumvent this problem by adding an empty use of \commandtt{mathop} in between.
We can do this by using \commandtt{DeclarePairedDelimiterXPP} to insert \commandtt{mathop\{\}} before the left delimiter:
\begin{showcode}{Defining~\commandname{newnorm} with~\commandtt{DeclarePairedDelimiterXPP}}
\DeclarePairedDelimiterXPP{\newnorm}[1]{\mathop{}}{\lVert}{\rVert}{}{#1}
\end{showcode}
The output now looks as follows:
\begin{showlatex}{Using \commandtt{newnorm}}
\[
  \newnorm{f}
  =
  \sup_{x \in I} \newnorm{f(x)}
\]
\end{showlatex}
This approach is motivated by \cite{tex_advancedpair}.

More information on \commandtt{DeclarePairedDelimiter} can be found in the \texttt{mathtools} manual.



\subsection{\texttt{xparse}}

A useful way for defining more involved macros is the \texttt{xparse} package.
This package provides the \commandtt{NewDocumentCommand} command, which can be used to define new commands.
The syntax of \commandtt{NewDocumentCommand} is as follows:
\begin{showcode}{Syntax of \commandtt{NewDocumentCommand}}
\NewDocumentCommand{\name}{arguments}{definition}
\end{showcode}
Instead of giving the number of arguments (as done for \commandtt{newcommand}) we say what kind of arguments will be given.
There are four possible kinds of argument:
\begin{itemize}
  \item
    \texttt{s} if a starred version of the argument should also be defined,
  \item
    \texttt{o} for optional arguments,
  \item
    \texttt{O} for optional arguments that admit a default value
  \item 
    \texttt{m} for mandatory arguments.
\end{itemize}
Mandatory arguments are given in curly brackets \texttt{\{ \}} and optional arguments are given in square brackets \texttt{[ ]}.
This is most easy to understand through some examples:

\subsubsection{First example}

\begin{showcode}{Using~\commandname{NewDocumentCommand}~I, defining~\commandname{restrict}}
\NewDocumentCommand{\restrict}{smm}{
  \IfBooleanTF{#1}
    {\left. {#2} \right|_{#3}}
    {#2|_{#3}}
}
\end{showcode}
% define the command
\NewDocumentCommand{\restrict}{smm}{
  \IfBooleanTF{#1}
    {\left. {#2} \right|_{#3}}
    {#2|_{#3}}
}%
The letter \texttt{s} ensures that the code defines both the command \commandtt{restrict} and its starred version \commandtt{restrict*}.
Both of these take two mandatory arguments.
The following part checks which version of the command is used:
\begin{showcode}{How to check for the star}
\IfBooleanTF{#1}{ starred command }{ non-starred command }
\end{showcode}
So \commandtt{restrict\{f\}\{X\}} gives the code \texttt{f|\_\{X\}} whereas \commandtt{restrict*\{f\}\{X\}} gives the following code:
\begin{center}
  \texttt{{\tbs}left. f {\tbs}right|\_\{X\}}
\end{center}
We get the following output:
\begin{showlatex}{Using \commandtt{restrict}}
\[
  \restrict{f}{X}
  \quad
  \restrict{\frac{f}{g}}{X}
  \quad
  \restrict*{\frac{f}{g}}{X}
\]
\end{showlatex}

\subsubsection{Second example}

The following is modified version of the previous example:
\begin{showcode}{Using~\commandname{NewDocumentCommand}~II, defining an enhanced version of~\commandname{restrict}}
\NewDocumentCommand{\restrict}{smmO{}}{
\IfBooleanTF{#1}
  {\left. {#2} \right|_{#3}^{#4}}
  {{#2} |_{#3}^{#4}}
}
\end{showcode}
% redefine the command
\RenewDocumentCommand{\restrict}{smmO{}}{
\IfBooleanTF{#1}
  {\left. {#2} \right|_{#3}^{#4}}
  {{#2} |_{#3}^{#4}}
}
The command now takes an optional third argument.
If this optional argument is not set then it takes on its standard value, which is empty.
\begin{showlatex}{Using the enhanced version of \commandtt{restrict}}
\[
  \restrict{f}{X}
  \quad
  \restrict{f}{X}[Y]
  \quad
  \restrict*{\frac{f}{g}}{X}
  \quad
  \restrict*{\frac{f}{g}}{X}[Y]
\]
\end{showlatex}

\subsubsection{Third example}

The following command takes two mandatory arguments and two optional arguments, whose standard values are empty:
\begin{showcode}{Using~\commandname{NewDocumentCommand}~III, defining~\commandname{moduleindex}}
\NewDocumentCommand{\moduleindex}{O{} m O{}}{
  {}_{#1} #2_{#3}
}
\end{showcode}
\NewDocumentCommand{\moduleindex}{O{} m O{}}{
  {}_{#1} #2_{#3}
}
The output is as follows:
\begin{showlatex}{Using \commandtt{moduleindex}}
\[
  \moduleindex{M}
  \quad
  \moduleindex[R]{M}
  \quad
  \moduleindex{M}[S]
  \quad
  \moduleindex[R]{M}[S]
\]
\end{showlatex}

\subsubsection{Fourth example}

The following example takes an optional argument with no standard value.
So it has to be checked if this optional argument was assigned a value.
\begin{showcode}{Using~\commandname{NewDocumentCommand}~IV, defining~\commandname{module}}
\NewDocumentCommand{\module}{m o}{
  \IfNoValueTF{#2}{{#1}-module}
                  {{#1}-{#2}-bimodule}
}
\end{showcode}
% overwrite the command
\NewDocumentCommand{\module}{m o}{%
  \IfNoValueTF{#2}{{#1}-module}
                  {{#1}-{#2}-bimodule}%
}
This command works as follows:
\begin{showlatex}{Using\commandtt{module}}
Let $M$ be an \module{$R$} and let $N$ be an \module{$R$}[$S$].
\end{showlatex}





\section{Don’t use \commandtt{limits}}

For the love of God, please don’t.
People seem to think that they have to add \commandtt{limits} after a command to add limits to it:
\begin{showlatex}*{Using limits}
\[
  \sum\limits_{k=1}^n k
  =
  \frac{n(n+1)}{2}
\]
\end{showlatex}
But this is not only unnecessary, but also dangerous.
It is unnecessary because (most of) the commands in questions already have this functionality built in:
\begin{showlatex}*{Using built-in limits}
\[
  \sum_{k=0}^n k
  =
  \frac{n(n+1)}{2}
\]
\end{showlatex}
This built-in limits have the advantage that they can distinguish between inline math and display math math:
\begin{showlatex}*{Inline vs.\ display with built-in limits}
The sum $\sum_{k=0}^n 2^k = 2^{n+1} - 1$ is inline while the sum
\[
  \sum_{k=0}^n 3^k
  \neq
  3^{n+1} - 1
\]
is in display mode.
\end{showlatex}
We can see that the inline version does not only use the smaller summation sign, but also sets the limits to the right instead of above the top and below the bottom of the summation sign.
This is a feature---a feature that the \commandtt{limits} version is missing:
\begin{showlatex}*{Inline with~\commandname{limits}}
So here is some text which will generate some lines.
The text itself isn’t important, but we really want it to fill some lines.
The important thing is the sum $\sum\limits_{k=0}^n k$.
Well, not really the sum itself, but its typesetting using the \commandtt{limits} command.
I think you know what I mean.
\end{showlatex}
The limits are still placed above the top and below the summation sign.
This has it’s price:
The line in which the sum resides breaks the usual vertical space between lines, which gives the text an inconsistent and unorganized look.
Compare this to the version without \commandtt{limits}:
\begin{showlatex}*{Inline with built-in limits}
So here is some text which will generate some lines.
The text itself isn’t important, but we really want it to fill some lines.
The important thing is the sum $\sum_{k=0}^n k$.
Well, not really the sum itself, but its typesetting without the \commandtt{limits} command.
\end{showlatex}
Here the line distance is nicely consistent and pleasing to the eye.

The usual predefined commands on which one would expect limits already have them defined, e.g.\ \commandtt{sum}, \commandtt{prod} or \commandtt{lim}, as the following example shows:
\begin{showlatex}{Inline vs.\ display for different commands with built-in limits}
Compare the inline versions $\sum_{k=0}^n k$ and $\prod_{k=1}^n k$ and $\lim_{n \to \infty} a_n$ with the display versions
\[
  \sum_{k=0}^n k \,,
  \quad
  \prod_{k=1}^n k \,,
  \quad
  \lim_{n \to \infty} a_n \,.
\]
\end{showlatex}
When one defines custom commands using \commandtt{DeclareMathOperator} and \commandtt{operatorname} one can make them support limits by using \commandtt{DeclareMathOperator*} and \commandtt{operatorname*} instead.
Suppose for examlp that we make in the preamble the following definition:
\begin{showcode}{Using \commandtt{DeclareMathOperator*} to define~\commandname{colim}}
\DeclareMathOperator*{\colim}{colim}
\end{showcode}
We can then do the following:
\begin{showlatex}{Using~\commandname{colim}}
Inline we have $\colim_{X' \leq X} F(X')$ and in display mode we get
\[
  \colim_{X' \leq X} F(X') \,.
\]
\end{showlatex}
The command \commandtt{operatorname*} was \commandtt{operatornamewithlimits} in the past, but this version should no longer be used.

If for some extremly strange reason one \emph{really} needs the limits to be in display style, then one should commit to it by using \commandtt{displaystyle}.
Consider the following example:
\begin{showlatex}{Forcing displaystyle with \commandtt{displaystyle}}
\[
  \begin{pmatrix}
    \frac{n^2+1}{n^2 + 2}
    &
    \frac{n^2+2}{n^2 + 3}
    \\
    \frac{n^2+2}{n^2 + 3}
    &
    \frac{n^2+3}{n^2 + 4}
  \end{pmatrix}
  =
  \begin{pmatrix}
    \displaystyle
    \frac{n^2+1}{n^2 + 2}
    &
    \displaystyle
    \frac{n^2+2}{n^2 + 3}
    \\[1.5em]
    \displaystyle
    \frac{n^2+2}{n^2 + 3}
    &
    \displaystyle
    \frac{n^2+3}{n^2 + 4}
  \end{pmatrix}
\]
\end{showlatex}
In the above example we have put~\commandname{displaystyle} before every matrix entry to ensure that it has the usual style of display mathematics.
We have also put replaced the basic line break~\commandname{\tbs} by~\inlinecode{{\tbs}{\tbs}[1.5em]} for some additional spacing between the two rows of the resulting matrix.
(Otherwise the fractions are far too close.)




\section{Use \texorpdfstring{\commandname{xrightarrow}}{{\tbs}xrightarrow} intstead of \texorpdfstring{\commandname{overset}}{{\tbs}overset}}
\label{extensible arrows}

Some people put text above arrows by wrongly using \commandname{overset} and \commandname{underset}:
\begin{showlatex}{Using~\commandname{overset} and~\commandname{underset} to put text above or below an arrow}
\[
  X
  \overset{f}{\longrightarrow}
  Y
  \underset{g \circ h \circ k}{\longrightarrow}
  Z
\]
\end{showlatex}
We can see above that the length of the arrow does not adjust to the size of the text above it or below it.
The proper way to put text on top of an arrow or bellow an arrow of the form~\enquote{$\to$} is therefore the command~\commandname{xrightarrow}:
\begin{showlatex}{Using~\commandname{xrightarrow} to put text above or below an arrow}
\[
  X
  \xrightarrow{f}
  Y
  \xrightarrow[g \circ h \circ k]{}
  Z \,.
\]
\end{showlatex}
The package~\packagename{amsmath} defines only the two most basic extensible arrows.
Many more kinds of extensible arrows are provided by the package~\packagename{mathtools} and some more are contained in the package~\packagename{extarrows}.
An overview of the various kinds of extensible arrows can be found in \cref{extensible arrow table}.
\begin{table}[tb]
  \begin{center}
  \begingroup
  \renewcommand{\arraystretch}{0.9}
  \begin{tabular}{@{}ccc@{}}
    \toprule
    \multicolumn{3}{c}{\textbf{extensible arrows in \packagename{amsmath}, \packagename{mathtools}, \packagename{extarrows}}}
    \\
    \midrule
    \begin{tabular}{c}
      $A \xrightarrow{f} B$
      \\
      \commandname{xrightarrow}
    \end{tabular}
    &
    \begin{tabular}{c}
      $A \xleftarrow{f} B$
      \\
      \commandname{xleftarrow}
    \end{tabular}
    &
    {}
    \\
    \midrule
    \begin{tabular}{c}
      $A \xmapsto{f} B$
      \\
      \commandname{xmapsto}
    \end{tabular}
    &
    \begin{tabular}{c}
      $A \xleftrightarrow{f} B$
      \\
      \commandname{xleftrightarrow}
    \end{tabular}
    &
    \begin{tabular}{c}
      $A \xRightarrow{f} B$
      \\
      \commandname{xRightarrow}
    \end{tabular}
    \\[1.5em]
    \begin{tabular}{c}
      $A \xLeftarrow{f} B$
      \\
      \commandname{xLeftarrow}
    \end{tabular}
    &
    \begin{tabular}{c}
      $A \xLeftrightarrow{f} B$
      \\
      \commandname{xLeftrightarrow}
    \end{tabular}
    &
    \begin{tabular}{c}
      $A \xhookleftarrow{f} B$
      \\
      \commandname{xhookleftarrow}
    \end{tabular}
    \\[1.5em]
    \begin{tabular}{c}
      $A \xhookrightarrow{f} B$
      \\
      \commandname{xhookrightarrow}
    \end{tabular}
    &
    \begin{tabular}{c}
      $A \xrightharpoondown{f} B$
      \\
      \commandname{xrightharpoondown}
    \end{tabular}
    &
    \begin{tabular}{c}
      $A \xrightharpoonup{f} B$
      \\
      \commandname{xrightharpoonup}
    \end{tabular}
    \\[1.5em]
    \begin{tabular}{c}
      $A \xrightleftharpoons{f} B$
      \\
      \commandname{xrightleftharpoons}
    \end{tabular}
    &
    \begin{tabular}{c}
      $A \xleftharpoondown{f} B$
      \\
      \commandname{xleftharpoondown}
    \end{tabular}
    &
    \begin{tabular}{c}
      $A \xleftharpoonup{f} B$
      \\
      \commandname{xleftharpoonup}
    \end{tabular}
    \\[1.5em]
    \begin{tabular}{c}
      $A \xleftrightharpoons{f} B$
      \\
      \commandname{xleftrightharpoons}
    \end{tabular}
    &
    {}
    &
    {}
    \\
    \midrule
    \begin{tabular}{c}
      $A \xlongequal{f} B$
      \\
      \commandname{xlongequal}
    \end{tabular}
    &
    \begin{tabular}{c}
      $A \xleftrightarrow{f} B$
      \\
      \commandname{xleftrightarrow}
    \end{tabular}
    &
    \begin{tabular}{c}
      $A \xLeftrightarrow{f} B$
      \\
      \commandname{xLeftrightarrow}
    \end{tabular}
    \\[1.5em]
    \begin{tabular}{c}
      $A \xlongleftarrow{f} B$
      \\
      \commandname{xlongleftarrow}
    \end{tabular}
    &
    \begin{tabular}{c}
      $A \xlongrightarrow{f} B$
      \\
      \commandname{xlongrightarrow}
    \end{tabular}
    &
    \begin{tabular}{c}
      $A \xlongleftrightarrow{f} B$
      \\
      \commandname{xlongleftrightarrow}
    \end{tabular}
    \\[1.5em]
    \begin{tabular}{c}
      $A \xLongleftarrow{f} B$
      \\
      \commandname{xLongleftarrow}
    \end{tabular}
    &
    \begin{tabular}{c}
      $A \xLongrightarrow{f} B$
      \\
      \commandname{xLongrightarrow}
    \end{tabular}
    &
    \begin{tabular}{c}
      $A \xLongleftrightarrow{f} B$
      \\
      \commandname{xLongleftrightarrow}
    \end{tabular}
    \\
    \bottomrule
  \end{tabular}
  \endgroup
  \end{center}
 \caption{Extensible arrows.}
 \label{extensible arrow table}
\end{table}

The author recommends to define custom commmands as shortcuts for the most used arrows.
\begin{showlatex}{Defining arrow commands as shortcuts}
\newcommand{\xto}{\xrightarrow}
\newcommand{\xlongto}[1]{\xlongrightarrow{\;#1\;}}
The function $A \xto{f} B$ is the same as
\[
  A \xlongto{f} B \,.
\]
\end{showlatex}

% TODO: Defining new kind of stretchable arrows





\section{Stretch your arrows}

If some expression occurs atop or below an arrow, then this arrow must be stretched suffciently long to accommodate these expression.
The extensible arrows introduced in \cref{extensible arrows} automatically do so.
If an arrow in a commutative diagram isn’t long enough then this arrow must also be stretched:
\begin{showlatex}{Commutative diagram with an arrow too short}
\[
\begin{tikzcd}
  X \arrow{r}{f \circ g - g \circ f} \arrow{d}
  &
  Y \arrow{r}{k} \arrow{d}
  &
  Z \arrow[equal]{d}
  \\
  X' \arrow[dashed]{r}{h'}
  &
  Y' \arrow{r}{k'}
  &
  Z'
\end{tikzcd}
\]
\end{showlatex}
By using the option~\inlinecode{column~sep~=~*} one can change the distance of the columns of the commutative diagram.
As options for~\inlinecode{*} one can give a specific distance, like~\inlinecode{4~em}.
Some distances are predefined for easy use, see \cref{column sep settings}.
\begin{table}[tb]
  \begin{center}
  \begin{tabular}{@{}lcccccc@{}}
    \toprule
    \textbf{name}
    &
    \inlinecode{tiny}
    &
    \inlinecode{small}
    &
    \inlinecode{scriptsize}
    &
    \inlinecode{normal}
    &
    \inlinecode{large}
    &
    \inlinecode{huge}
    \\
    \textbf{distance}
    &
    \inlinecode{0.6em}
    &
    \inlinecode{1.2em}
    &
    \inlinecode{1.8em}
    &
    \inlinecode{2.4em}
    &
    \inlinecode{3.6em}
    &
    \inlinecode{4.8em}
    \\
    \bottomrule
  \end{tabular}
  \end{center}
  \caption{Standard distances for~\inlinecode{column sep}.}
  \label{column sep settings}
\end{table}
With~\inlinecode{column~sep} one can fix the above diagram.
\begin{showlatex}{Using~\inlinecode{column sep}}
\[
\begin{tikzcd}[column sep = huge]
  X \arrow{r}{f \circ g - g \circ f} \arrow{d}
  &
  Y \arrow{r}{k} \arrow{d}
  &
  Z \arrow[equal]{d}
  \\
  X' \arrow[dashed]{r}{h'}
  &
  Y' \arrow{r}{k'}
  &
  Z'
\end{tikzcd}
\]
\end{showlatex}
% TODO: Change the distance between only specific columns.

One can similarly use the option~\inlinecode{row~sep~=~*} to change the distance of the rows.
Some predefined distances can be found in \cref{row sep settings}.
\begin{table}[tb]
  \begin{center}
  \begin{tabular}{@{}lcccccc@{}}
    \toprule
    \textbf{name}
    &
    \inlinecode{tiny}
    &
    \inlinecode{small}
    &
    \inlinecode{scriptsize}
    &
    \inlinecode{normal}
    &
    \inlinecode{large}
    &
    \inlinecode{huge}
    \\
    \textbf{distance}
    &
    \inlinecode{0.45em}
    &
    \inlinecode{0.9em}
    &
    \inlinecode{1.35em}
    &
    \inlinecode{1.8em}
    &
    \inlinecode{2.7em}
    &
    \inlinecode{3.6em}
    \\
    \bottomrule
  \end{tabular}
  \end{center}
  \caption{Standard distances for~\inlinecode{row sep}.}
  \label{row sep settings}
\end{table}
These are the same as for~\inlinecode{column~sep} but scaled down by a factor of~$0.75$.





\section{Know your dots}

Ellipsis should \textbf{never} by written as \texttt{...} (three single dots), see the following horrible example:
\begin{showlatex}*{Very wrong ellipsis with~\inlinecode{...}}
$a + ... + b$
\end{showlatex}
\hologo{LaTeX} instead provides different kinds of ellipses to use in math mode, namely
\begin{center}
    \commandtt{dotsb},
    \quad
    \commandtt{dotsc},
    \quad
    \commandtt{dotsm},
    \quad
    \commandtt{dotsi},
    \quad
    \commandtt{dotso},
    \\
    \commandtt{cdots},
    \quad
    \commandtt{ddots},
    \quad
    \commandtt{vdots},
    \quad
    \commandtt{ldots}.
\end{center}
Each of these have they own role and (mostly) distinct look and feel.
There are two groups of ellipses.
The first group consists of \commandtt{dotsb}, \commandtt{dotsc}, \commandtt{dotsm}, \commandtt{dotsi},\commandtt{dotso}, whereas the second group consists of \commandtt{cdots}, \commandtt{ddots}, \commandtt{vdots}, \commandtt{ldots}.

The ellipses in the first group are named after they function:
The ellipsis \commandtt{dotsb} for binary, \commandtt{dotsc} for comma, \commandtt{dotsm} for multiplication, \commandtt{dotsi} for integrals and \commandtt{dotso} for other.
The usage and look of the first three are as follows:
\begin{showlatex}*{Using~\commandname{dotsb},~\commandname{dotsc},~\commandname{dotsi},~\commandname{dotsm}}
\begin{gather*}
  \sum_{i=1}^n x_i
  =
  x_1 + \dotsb + x_n 
  \\
  1 \leq \dotsb \leq n 
  \\
  i = 1, \dotsc, m
  \\
  \int_{a_1}^{b_1}
  \dotsi
  \int_{a_n}^{b_n}
  \\
  \prod_{i=1}^n x_i
  =
  x_1 \dotsm x_n
\end{gather*}
\end{showlatex}
The command \commandtt{dotso} is meant for other, more unspecific cases.
Note that the members of this group all have names of the form \commandtt{dots*}, where \texttt{*} is a letter that specifies the semantic use of the dots.

The ellipses in the second group are not named after their function but after their look:
The ellipsis \commandtt{cdots} is vertically centered, \commandtt{ddots} is diagonal (from the top left to the bottom right), \commandtt{vdots} is horizontally centered, and \commandtt{ldots} is vertically lowered.
The first three of these ellipses are usually used in matrices:
\begin{showlatex}{Using~\commandname{cdots},~\commandname{vdots},~\commandname{ddots}}
\[
  \begin{pmatrix}
    a_{11} & \cdots & a_{1n} \\
    \vdots & \ddots & \vdots \\
    a_{n1} & \cdots & a_{nn}
  \end{pmatrix}
\]
\end{showlatex}
The command \commandtt{ldots} can for example be used to denote left out digits or symbols.
\begin{showlatex}{Using~\commandtt{ldots}}
We find that
\[
  x = 0.1234567891011\ldots
\]
Consider the word $w = a_1 \ldots a_n$ where $a_1, \dotsc, a_n$ are letters in an alphabet $\Sigma$.
\end{showlatex}
Note that the members of this second group of dots all have names of the form \texttt{*dots}, where \texttt{*} is a letter that specified the positioning of these dots.

The following should \textbf{never} to denote a product:
\begin{showcode}{Wrong way of abbreviating multiplication~I}
x_1 \cdot {any kind of dots} \cdot x_n
\end{showcode}
So all of the following are \textbf{wrong}, and some of them look even worse then the other ones.
\begin{showlatex}*{Wrong way of abbreviating multiplication~II}
\begin{gather*}
  x_1 \cdot \dotsb \cdot x_n \\
  x_1 \cdot \dotsc \cdot x_n \\
  x_1 \cdot \dotsm \cdot x_n \\
  x_1 \cdot \dotsi \cdot x_n \\
  x_1 \cdot \cdots \cdot x_n \\
  x_1 \cdot \ldots \cdot x_n
\end{gather*}
\end{showlatex}
(The author hopes that nobody is stupid enough to even trying using \commandtt{ddots} or \commandtt{vdots} in this situation.)

So overall one should use \commandtt{cdots}, \commandtt{ddots} and \commandtt{vdots} for matrices, and otherwise the commands \commandtt{dotsb}, \commandtt{dotsc}, \commandtt{dotsm}, \commandtt{dotsi} or \commandtt{dotso} depending on the context.

There also exists the generic command \commandtt{dots} which tries to automagically use the right kind of positioning and spacing.





\section{Know your matrices}

The following are the usual kinds of matrices.
\begin{showlatex}{Different kinds of matrices}
\begin{gather*}
  \begin{matrix}
    a & b \\
    c & d
  \end{matrix}
  \quad
  \begin{pmatrix}
    a & b \\
    c & d
  \end{pmatrix}
  \quad
  \begin{bmatrix}
    a & b \\
    c & d
  \end{bmatrix}
  \quad
  \begin{vmatrix}
    a & b \\
    c & d
  \end{vmatrix}
\end{gather*}
\end{showlatex}
There alse exist starred versions of these matrix environments that allow to specify the alignment of its columns.
(The standard is centered.)
\begin{showlatex}{Aligning matrix entries}
\[
  \begin{matrix*}[r]
      a & -b \\
     -c &  d
  \end{matrix*}
  \qquad
  \begin{pmatrix*}[c]
    a & -b \\
    -c &  d
  \end{pmatrix*}
  \qquad
  \begin{bmatrix*}[l]
     a & -b \\
    -c &  d
  \end{bmatrix*}
  \qquad
  \begin{vmatrix}
     a & -b \\
    -c &  d
  \end{vmatrix}
\]
\end{showlatex}
There are also small versions of all of the above:
\begin{showlatex}{Different kind of (aligned) small matrices}
\begin{gather*}
  \begin{smallmatrix}
    a & b \\
    c & d
  \end{smallmatrix}
  \qquad
  \begin{psmallmatrix}
    a & b \\
    c & d
  \end{psmallmatrix}
  \qquad
  \begin{bsmallmatrix}
    a & b \\
    c & d
  \end{bsmallmatrix}
  \qquad
  \begin{vsmallmatrix}
    a & b \\
    c & d
  \end{vsmallmatrix}
\\
  \begin{smallmatrix*}[r]
    -a &  b \\
      c & -d
  \end{smallmatrix*}
  \quad
  \begin{psmallmatrix*}[c]
    -a &  b \\
      c & -d
  \end{psmallmatrix*}
  \quad
  \begin{bsmallmatrix*}[l]
    -a &  b \\
      c & -d
  \end{bsmallmatrix*}
  \quad
  \begin{vsmallmatrix}
    -a &  b \\
      c & -d
  \end{vsmallmatrix}
\end{gather*}
\end{showlatex}
One needs to choose the type of matrix depending on the context:
\begin{showlatex}*{Using the right kind of matrix}
We consider the morphism
\[
  X \oplus Y
  \xrightarrow{\,
    \begin{bsmallmatrix*}[r]
      f & -d \\
      0 &  g
    \end{bsmallmatrix*}
    \,
  }
  X' \oplus Y' \,.
\]
\end{showlatex}





\section{Use \texorpdfstring{$\mathrm{d}x$}{dx} instead of \texorpdfstring{$dx$}{dx}}

A differential is written as \texttt{{\tbs}mathrm\{d\}x}, not as $dx$.
When it occurs at the end of an integral then a slight spacing \texttt{{\tbs},} is also introduced.
The following are therefore \textbf{wrong}:
\begin{showlatex}*{Wrong kind of differentials}
\[
  \int_a^b f(x) dx
  \quad
  \int_a^b f(x) \mathrm{d}x
\]
\end{showlatex}
The following is right:
\begin{showlatex}*{Right kind of differential}
\[
  \int_a^b f(x) \,\mathrm{d}x
\]
\end{showlatex}




