\chapter{Other things}



% \section{Don’t use inline for long formulas}
% 
% The change in \cref{breaking inline math} tends to lead to problems with long formulas or equations, as the surrounding text has to be arranged in a way that they are contained in a single line.
% This is a feature of \cref{breaking inline math}:
% If formulas and equations are too long then they should be put into display mode, not inline mode!
% Compare the following two examples:
% \begin{LTXexample}[pos = b]
%   Suppose that you have already written a bunch of text.
%   Now you start talking about the inequality $\lcm([L_1 : K],[L_2 : K]) \leq [L_1 L_2 : K] \leq [L_1 : K] [L_2 : K]$,
%   which is a bit long.
%   The surrounding text doesn’t really help when you want to focus on the formula.
% \end{LTXexample}
% \begin{LTXexample}[pos = b]
%   Suppose that you have already written a bunch of text.
%   Now you start talking about the inequality
%   \[
%           \lcm([L_1 : K],[L_2 : K])
%     \leq  [L_1 L_2 : K]
%     \leq  [L_1 : K] [L_2 : K]
%   \]
%   which is a bit long.
%   Now the surrounding text doesn’t matter when you want to focus on the formula.
% \end{LTXexample}





% \section{Put important things in display mode}
% 
% Putting mathematical content in display mode distinguishes it from the surrounding text.
% This can be used to emphasize its importance.
% It therefore makes sense to put certain contents into display mode even though it’s short and can reasonable fit into inline mode.
% 
% \begin{LTXexample}[pos = b]
%   If $I$ is an ideal in a commutative ring $R$ and $M$ is an $R$-module then
%   \[
%     (R/I) \otimes_R M
%     \cong
%     M / IM \,.
%   \]
%   This will turn out to be a rather useful identity.
% \end{LTXexample}


