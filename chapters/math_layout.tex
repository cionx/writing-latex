\chapter{Mathematical layout}





\section{Don’t use \texttt{\$\$  \$\$} or \texttt{eqnarray}}

There are many good ways to put mathematical content into display mode, but \texttt{\$\$  \$\$} and \texttt{eqnarray} are none of them.
The method \texttt{\$\$  \$\$} is too level for practical use, and \texttt{eqnarray} has too many problems and has been deprecated since forever.
One can use any of the following, depending on the planned usage.



\subsection{\texttt{{\tbs}[ {\tbs}]} and \texttt{equation*}}

The methods \texttt{equation*} and \texttt{{\tbs}[ {\tbs}]} can be used for a single line of display math mode.
Both commands do the same thing (when \texttt{amsmath} is loaded).
\begin{showlatex}*{Using \texttt{{\tbs}[ {\tbs}]} and \texttt{equation*}}
Suppose that both the formula
\[
  a + b = c
\]
and the formula
\begin{equation*}
  2a - b = c \,.
\end{equation*}
hold.
Then $a$ and $b$ are unique.
\end{showlatex}
The non-starred version \texttt{equation} numbers the equation.
\begin{showlatex}*{Using \texttt{equation}}
The formula
\begin{equation}
  a^2 - b^2 = (a + b)(a - b)
\end{equation}
is one of the binomial formulas.
\end{showlatex}



\subsection{\texttt{gather*}}

The \texttt{gather*} environment is meant for multiple lines that are non-aligned but centered instead.
\begin{showlatex}{Using \texttt{gather*}}
We consider for every integer $n \geq 0$ the polynomial
\[
  p_n
  =
  \sum_{k=0}^n x^k \,.
\]
In particular
\begin{gather*}
  p_0 = 1 \,,
  \qquad
  p_1 = 1 + x \,,
  \qquad
  p_2 = 1 + x + x^2 \,,
  \qquad
  p_3 = 1 + x + x^2 + x^3 \,,
  \\
  p_4 = 1 + x + x^2 + x^3 + x^4 \,,
  \qquad
  p_5 = 1 + x + x^2 + x^3 + x^4 + x^5 \,.
\end{gather*}
\end{showlatex}
The non-starred version \texttt{gather} number the lines.
\begin{showlatex}{Using \texttt{gather}}
We have the polynomials
\begin{gather}
  p_0 = 1 \,,
  \qquad
  p_1 = 1 + x \,,
  \qquad
  p_2 = 1 + x + x^2 \,,
  \qquad
  p_3 = 1 + x + x^2 + x^3 \,,
  \\
  p_4 = 1 + x + x^2 + x^3 + x^4 \,,
  \qquad
  p_5 = 1 + x + x^2 + x^3 + x^4 + x^5 \,.
\end{gather}
\end{showlatex}



\subsection{\texttt{align*} and \texttt{alignat*}}

The \texttt{align*} enviroment takes multiple lines which can aligned.
Each line contains the symbol \texttt{\&} once, and the occurences of this symbol are then aligned.
\begin{showlatex}*{Using \texttt{align*}}
We find that
\begin{align*}
  a + b + c
  &=
  d + e + f + g
  \\
  &=
  h + i + j
  \\
  &=
  k + l + m + n \,.
\end{align*}
\end{showlatex}
The unstarred version \texttt{align} again numbers the lines.
\begin{showlatex}*{Using \texttt{align}}
We find again that
\begin{align}
  a + b + c
  &=
  d + e + f + g
  \\
  &=
  h + i + j
  \\
  &=
  k + l + m + n \,.
\end{align}
\end{showlatex}
One can also use multiple aligned columns, which then need to be separated by an additional \texttt{\&}.
For~$n$ aligned columns we hence need~$2n+1$ occurences of \texttt{\&}.
\begin{showlatex}{Using \texttt{align*} with multiple columns}
We consider the values
\begin{align*}
  x_1 &= 1 \,,  &   x_2 &= 2 \,,  &   x_3 &= 3 \,,  \\
  x_4 &= 4 \,,  &   x_5 &= 5 \,,  &   x_6 &= 6 \,,  \\
  x_7 &= 7 \,,  &   x_8 &= 8 \,,  &   x_9 &= 9 \,.
\end{align*}
\end{showlatex}

The \texttt{alignat*} environment is similar to the \texttt{align} environment, but doesn’t add any built-in spacing between the aligned columns.
Any such spacing must therefore by added by hand.
One also has to specify the number of columns beforehand.
\begin{showlatex}{Using \texttt{alignat*}}
We also consider the values
\begin{alignat*}{3}
  y_1 &= 9 \,,  &\qquad   y_2 &= 8 \,,  &\qquad   y_3 &= 7 \,,  \\
  y_4 &= 6 \,,  &         y_5 &= 5 \,,  &         y_6 &= 4 \,,  \\
  y_7 &= 3 \,,  &         y_8 &= 2 \,,  &         y_9 &= 1 \,.
\end{alignat*}
\end{showlatex}
When aligning multiple columns one should use \texttt{alignat*} instead of \texttt{align*} to get (manually) a good looking distance between the aligned columns.



\subsection{Don’t do \texttt{center} with \texttt{\$ \$}}

By all that is holy, don’t do the following:
\begin{showlatex}*{Using~\commandname{center} with~\inlinecode{\$ \$}}
Some text that ends in
\begin{center}
  $a + b = c$.
\end{center}
Now some more text.
\end{showlatex}
% TODO: Explain all the things that go wrong



\subsection{\texttt{gathered}, \texttt{aligned} and \texttt{alignedat}}

The environments \texttt{gathered}, \texttt{aligned} and \texttt{alignedat} are variations of the environments \texttt{gather*}, \texttt{align*} and \texttt{alignat*}.
These variations are meant to be used inside an already existing math environment.
\begin{showlatex}*{Using \texttt{aligned}}
\[
  \left\{
    \begin{aligned}
      a + b &= c      \\
      d     &= e + f
    \end{aligned}
  \right\}
\]
\end{showlatex}



\subsection{Overview}

One should always use the most basic environment that does the job:
Using overpowered environments can lead to unexpected problems.
Consider the following example:
\begin{showlatex}{Inproper use of \texttt{align*}}
\begin{align*}
  ABCD = EF = GHI = JKL = M = NOP
  \\
  QRS = TUV = WX = Y = Z
\end{align*}
\end{showlatex}
The \texttt{align*} environment automatically alignes both lines on the right since no information about alignment was given.
In the above situation one should use \texttt{gather*} instead.
\begin{showlatex}{Using \texttt{gather*} for non-aligned equations}
\begin{gather*}
  ABCD = EF = GHI = JKL = M = NOP
  \\
  QRS = TUV = WX = Y = Z
\end{gather*}
\end{showlatex}
The flowchart in \cref{environment flow chart} explains how to choose the correct display environment.
(This flowchart is partly inspired by \cite{flowchart}.)
\begin{figure}[tb]
  \begin{center}
  \begin{tikzpicture}[
    node distance = 5em,
    >={Latex[width=2mm,length=2mm]},
    every text node part/.style = { align = center },
    start/.style    = { rectangle,
                        rounded corners,
                        fill = green!20!white,
                        draw = black,
                        minimum width  = 5em,
                        minimum height = 2em
                      },
    question/.style = { rectangle,
                        rounded corners,
                        fill = black!10!white,
                        draw = black,
                        minimum width  = 9em,
                        minimum height = 2.5em
                      },
    answer/.style   = { rectangle,
                        rounded corners,
                        draw = black,
                        minimum width  = 7em,
                        minimum height = 1.5em
                      }
  ]
    % nodes
    \node (start)
          [start]
          {formula};
    \node (multiline)
          [question, below of = start]
          {multiline?};
    \node (alignment)
          [question, below of = multiline]
          {alignment?};
    \node (multicolumn)
          [question, below of = alignment]
          {multiple columns?};
    \node (equation)
          [answer, right of = multiline, xshift = 7em]
          {\texttt{{\tbs}[ {\tbs}]}\\equation*};
    \node (gather)
          [answer, right of = alignment, xshift = 7em]
          {gather*};
    \node (align)
          [answer, right of = multicolumn, xshift = 7em]
          {align*};
    \node (alignat)
          [answer, below of = align]
          {alignat*};
    % arrows
    \draw[->] (start) -- (multiline);
    \draw[->] (multiline)   -- node[anchor=south] {no} (equation);
    \draw[->] (multiline)   -- node[anchor=east] {yes} (alignment);
    \draw[->] (alignment)   -- node[anchor=south] {no} (gather);
    \draw[->] (alignment)   -- node[anchor=east] {yes} (multicolumn);
    \draw[->] (multicolumn) -- node[anchor=south] {no} (align);
    \draw[->] (multicolumn) |- node[anchor=east, yshift=1.7em] {yes} (alignat);
  \end{tikzpicture}
  \end{center}
  If numbering of the line(s) is needed then the unstarred version is to be used.
  \caption{Deciding on a math environment.}
  \label{environment flow chart}
\end{figure}





\section{Break and align at good points}
\label{break and align ponts}

Often times a formula is broken among multiple lines.
This is done for at least two reasons:
\begin{itemize}
  \item
    To improve the readability of the given formula.
  \item
    To prevent that the formula goes over the margins of the document.
\end{itemize}
If a formula is broken amoung multiple lines then one has to choose at which places the formla is broken, and how the resulting parts of the formula will be aligned.
In this section we present some standard ways of breaking and aligning a formula, and discuss when to break a formula.



\subsection{When to break}

We first discuss when a formula needs to be broken.

\subsubsection{When the formula is too long}

If a formula is too long to physically fit into the text area, i.e.\ if it goes over the margins of the text area, then it must be broken up.
These occurences are easy to spot since they give will give an \enquote{overfull hbox} warning.
Consider the following example:
\begingroup
\begin{showlatex}[before lower = {\hfuzz = 40pt}, after lower = {\hfuzz = 0pt}]{Overfull hbox}
  It follows from
  \[
    aaaaaaa
    =
    bbbbbbb
    =
    ccccccc
    \leq
    ddddddd
    =
    eeeeeee
    =
    fffffff
    \leq
    ggggggg
    =
    hhhhhhh
  \]
  that $a \leq h$.
\end{showlatex}
\endgroup

\subsubsection{When the formula is visually too long}

Sometimes a formula does fit into a single line, but barely so.
Consider the following example:
\begin{showlatex}{Visually overfull hbox}
Here is some text.
\[
  aaaaaaaaa = bbbbbbbbbbbbbb = cccccccccccccc = dddddddddddddd = eeeeeeeeee
\]
Here is some more text.
\end{showlatex}
The formula in the above example does actually not go over the margin.
But it has stretched the displaymode over its visual limits.
The formula does therefore neeed to be broken up.

\subsection{Expressing structure}

Often the breaking up of a formula is done to better express the structure---and thus content---of the displayed formula.
Consider the following example:
\begin{showlatex}[label={unreadable formula}]{A formula that should be broken up for readability}
If $k$ is algebraically closed and $i$ is a square root of $-1$ then
\[
  k[x]/(x^2 + 1)
  =
  k[x]/( (x - i) (x + i) )
  \cong
  k[x]/(x - i) \times k[x]/(x + i)
  \cong
  k \times k
\]
by the Chinese remainder theorem.
\end{showlatex}
The above output still has an appropriate length to be put into a single line, and if space is spare then this is an acceptable solution.
But this single-line approach to the formula does not help to display its internal structure.
This can be done by splitting up the formula as done in the next example:
\begin{showlatex}{Broken up version of \cref*{unreadable formula}}
If $k$ is algebraically closed and $i$ is a square root of $-1$ then
\begin{align*}
  k[x]/(x^2 + 1)
  &=
  k[x]/( (x - i) (x + i) )
  \\
  &\cong
  k[x]/(x - i) \times \mathbb{C}[x]/(x + i)
  \\
  &\cong
  k \times k
\end{align*}
by the Chinese remainder theorem.
\end{showlatex}
This form makes it clear where the equalities and isomorphisms occur.



\subsection{Where to break and align}

We now discuss at which points a formula can be broken, and how these broken parts can then be aligned.

\subsubsection{Aligning at relation symbols I}

A first approach is to all relation symbols under each other, 
\begin{showlatex}{Aligning relation symbols~I}
Let $x$,$y$ be two nilpotent elements in a commutative ring $R$.
Then
\begin{align*}
  \exp(x) \exp(y)
  &=
  \left( \sum_{k=0}^\infty \frac{x^k}{k!} \right)
  \left( \sum_{l=0}^\infty \frac{y^l}{l!} \right)
  \\
  &=
  \sum_{k,l=0}^\infty \frac{x^k y^l}{k! \, l!}
  \\
  &=
  \sum_{n=0}^\infty \, \sum_{k+l = n} \frac{x^k y^l}{k! \, l!}
  \\
  &=
  \sum_{n=0}^\infty \frac{1}{n!} \sum_{k=0}^n \binom{n}{k} x^k y^{n-k}
  \\
  &=
  \sum_{n=0}^\infty \frac{1}{n!} (x + y)^n
  \\
  &=
  \exp(x + y) \,.
\end{align*}
\end{showlatex}

\subsubsection{Aligning at relation symbols II}

One can also align all relation symbols to the left, so that the broken up parts of the formla all lie on top of each other.
\begin{showlatex}{Aligning relation symbols~II}
Let~$R$ and~$S$ be two commutative rings.
Then for every~$(r,s) \in R \times S$,
\begin{align*}
  {}&
  (r, s) \in (R \times S)^\times
  \\
  \iff{}&
  \text{there exist $(r', s') \in R \times S$ with $(r,s)(r',s') = (1,1)$}
  \\
  \iff{}&
  \text{there exist $r' \in R$ and $s' \in S$ with $rr' = r$ and $ss' = 1$}
  \\
  \iff{}&
  \text{$r \in R^\times$ and $s \in S^\times$}
  \\
  \iff{}&
  (r,s) \in R^\times \times S^\times \,,
\end{align*}
and therefore $(R \times S)^{\times} = R^\times \times S^\times$.
\end{showlatex}
The empty pair of curly brackets in \texttt{{\tbs}iff\{\}\&} ensures that the spacings coming from \commandtt{iff} and \texttt{\&} do not interfere with each other.
Otherwise something like this happens:
\begin{showlatex}{Wrong spacing when alignment points are set wrong}
  \begin{align*}
      &\text{some stuff} \\
    =&\text{some other stuff}
  \end{align*}
\end{showlatex}

\subsubsection{Breaking at a binary operator}

Sometimes it is also useful to break a long term of a formula at a binary operator.
In this case this operator needs to occur in the line after the break.
The following example does it wrong:
\begin{showlatex}*{Wrong aligning at a binary operator~I}
\begin{align*}
  & a + b + c + \\
  & d + e
\end{align*}
\end{showlatex}
The following should be done instead:
\begin{showlatex}*{Right breaking at a binary operator~I}
\begin{align*}
  & a + b + c \\
  & + d + e
\end{align*}
\end{showlatex}
If a formula is broken at relation symbols and one of the resulting terms is broken at a binary operator, then the operator is not aligned together with the relation symbols.
Instead the binary operator appears after the relation symbols.
The following example does it wrong:
\begin{showlatex}*{Wrong breaking at a binary operator~II}
\begin{align*}
  a + a
  &=
  b + b + b + b
  \\
  &+
  b + b + b + b
  \\
  &=
  c + c + c + c + c
\end{align*}
\end{showlatex}
Instead the following has to be done:
\begin{showlatex}*{Right breaking at a binary operator~II}
\begin{align*}
  a + a
  ={}&
  b + b + b + b
  \\
  {}&
  +b + b + b + b
  \\
  ={}&
  c + c + c + c + c
\end{align*}
\end{showlatex}

\subsubsection{A single term in another line}

If a single line equation is too long, then it is sometimes appropriate to put the last term in a new line, such that the last term occurs below the second to last term.
\begin{showlatex}{Single term in new line}
It follows that
\begin{align*}
  aaaaaaaaaaaa
  =
  bbbbbbbbbbb
  =
  cccccccccc
  =
  ddddddddd
  &=
  eeeeeeee
  \\
  &=
  ffffff
\end{align*}
and hence $2 + 2 = 5$.
\end{showlatex}



\subsection{Don’t break formulas badly}

One should always keep in mind that breaking up a formula isn’t just meant to prevent technical problems, but more importantly to let the resulting output display the structure---and thus part of the content---of the formula.
A badly broken up formula is harder to understand for both the reader and the author.
Consider the following example:
\begin{showlatex}{Badly broken formula}
It follows that
\begin{align*}
  aaaaaaaaa
  &=
  bbbbbbbbbbb
  =
  cccccccc
  \leq
  dddddd
  =
  eeeeeee
  \\
  &=
  eee
  \leq
  fffffff
  =
  gggggggg
  \leq
  hhhhhhhhh
  \\
  &<
  kkkkkkk
  =
  llll
  \leq
  mmmma
  =
  nnnnnnn
  =
  pp \,.
\end{align*}
\end{showlatex}
In such a case the breaking of the formula should be done in a consistent way.
There seeems to be two sensible approaches in this example:

\subsubsection{Align everything}

One can align all occuring relation symbols:
\begin{showlatex}{Aligning all relation symbols}
It follows that
\begin{align*}
  aaaaaaaaa
  &= bbbbbbbbbbb \\
  &= cccccccc \\
  &\leq dddddd \\
  &= eeeeeee \\
  &= eee \\
  &\leq fffffff \\
  &= gggggggg \\
  &\leq hhhhhhhhh \\
  &< kkkkkkk \\
  &= llll \\
  &\leq mmmma \\
  &= nnnnnnn \\
  &= pp \,.
\end{align*}
\end{showlatex}
This approach has the advantage of being very consistent.
But it has the disadvantage of taking a lot of space.
It also may not reflect the structure of the formula particularly well, as this layout gives all (in)equalities the same importance.
    
\subsubsection{Align at inequalities}

On could align all the inequality symbols, to make it clear where these occur:
\begin{showlatex}{Aligning all inequalities}
It follows that
\begin{align*}
  aaaaaaaaa
  &= bbbbbbbbbbb
  = cccccccc
  \\
  &\leq
  dddddd
  = eeeeeee
  = eee
  \\
  &\leq
  fffffff
  = gggggggg
  \\
  &\leq
  hhhhhhhhh
  \\
  &<
  kkkkkkk
  = llll \\
  &\leq mmmma
  = nnnnnnn
  \\
  &= pp \,.
\end{align*}
\end{showlatex}
This layout emphasizes the importance of the inequalities, while relegating the equalities to a less imporant position.
Note that we have also aligned the first and last equality signs to make it clear where the manipulations begin and end.
If some other equalities are also particularly important (e.g.\ if they follows from some previously hard-earned proposition) then they too should be alinged





\section{Use \commandtt{intertext} and \commandtt{shortintertext}}

Multiline environments like \texttt{gather*} and \texttt{align*} can be interrupted by using \commandtt{intersect} and \commandtt{shortintertext} to insert some text between different lines of mathematics.
This is particularly useful to combine two \texttt{align*} environments into a single \texttt{align*} environment, which allows for a common alignment of both sections.

The following is an example for what we don’t want.
\begin{showlatex}*{Two non-aligned blocks}
We consider the equalities
\begin{align*}
  H
  &=
  a_1 + a_2 + a_3 + a_4
  \\
  &=
  b_1 + b_2 + b_3 + b_4 + b_5 + b_6
  \\
  &=
  c_1 + c_2 + c_3 + c_4 + c_5
\end{align*}
and
\begin{align*}
  I
  &=
  d_1 + d_2 + d_3 + d_4 + d_5 + d_6
  \\
  &=
  e_1 + e_2 + e_3
  \\
  &=
  f_1 + f_2 + f_3 + f_4 + f_5 \,.
\end{align*}
\end{showlatex}
Note the annoying misalignment of the equality signs of the two \texttt{align*} environments.
We can solve this problem by using only one \texttt{align*} environment and inserting the text \enquote{and} by using the \commandtt{intertext} command:
\begin{showlatex}*{Two aligned blocks}
We consider now the equalities
\begin{align*}
  H
  &=
  a_1 + a_2 + a_3 + a_4
  \\
  &=
  b_1 + b_2 + b_3 + b_4 + b_5 + b_6
  \\
  &=
  c_1 + c_2 + c_3 + c_4 + c_5
\intertext{and}
  I
  &=
  d_1 + d_2 + d_3 + d_4 + d_5 + d_6
  \\
  &=
  e_1 + e_2 + e_3
  \\
  &=
  f_1 + f_2 + f_3 + f_4 + f_5 \,.
\end{align*}
\end{showlatex}
The \texttt{shortintertext} functions in the same way as \texttt{intertext} but inserts less space between the two equations.
To see the effects of \texttt{shortintertext} let us start with the following example:
\begin{showlatex}{Two non-aligned equalities}
First we have some lengthy text above the matical formulas, followed by the formulas
\[
  a = b
\]
and
\[
  b = a \,.
\]
We finish off by writing some more text after the above mathematical two formulas.
\end{showlatex}
We naturally want to align the two equality signs, which we by using a single \texttt{align*} environment together with \texttt{intertext}, as explained above:
\begin{showlatex}{Two aligned equalities with too much space}
First we have some lengthy text above the matical formulas, followed by the formulas
\begin{align*}
  a &= b
\intertext{and}
  b &= a \,.
\end{align*}
We finish off by writing some more text after the above mathematical two formulas.
\end{showlatex}
But we can seet that the inserted \texttt{intersect} puts a large space between the two formulas, much too large for our taste.
We can solve this problem by using \texttt{shortintertext} instead of \texttt{intertext}, as the following example shows:
\begin{showlatex}{Two aligned equalities with proper space}
First we have some lengthy text above the matical formulas, followed by the formulas
\begin{align*}
  a &= b
\shortintertext{and}
  b &= a \,.
\end{align*}
We finish off by writing some more text after the above mathematical two formulas.
\end{showlatex}





\section{Uses \texorpdfstring{\envname{cases}}{cases}}

Use the environment \envname{cases} for case distinctions:
\begin{showlatex}{Using \envname{cases}}
It follows that
\[
  A(x)
  =
  \begin{cases}
    x^2  & \text{if $x \leq 0$,} \\
    3x   & \text{if $x = 0$.}
  \end{cases}
\]
\end{showlatex}
In most cases one should actually use the environment \envname{cases*}, which ensures that the second column will be treated as text.
\begin{showlatex}{Using \envname{cases*}}
It follows that
\[
  A(x)
  =
  \begin{cases*}
    x^2  & if $x \leq 0$, \\
    3x   & if $x = 0$.
  \end{cases*}
\]
\end{showlatex}





\section{Aligning nearly alinged formulas}

Sometimes formulas turn out to look nearly aligned in the compiled output, even though this wasn’t planned.
But the formulas may still be non-aligned enough to look jarring.
In such a case it is often best to align these formulas.

Consider the following example:
\begin{showlatex}{Accidental jarringly non-alined expressions}
\begin{gather*}
  KK^{-1} = 1 = K^{-1}K \,,
  \quad
  EF - FE = \frac{ K - K^{-1} }{ q - q^{-1} } \,,
  \\
  KE = q^2 EK \,,
  \quad
  KF = q^{-2} FK \,.
\end{gather*}
\end{showlatex}
Note that the first line of the output looks slightly shifted to the left when compared to the second line.
But properly aligning both lines this impression vanishes and instead gives rise to more coherent look and fell.
\begin{showlatex}{Intentional well-aligned expressions}
\begin{align*}
  KK^{-1} = 1 = K^{-1}K \,,
  \quad
  &EF - FE = \frac{ K - K^{-1} }{ q - q^{-1} } \,,
  \\
  KE = q^2 EK \,,
  \quad
  &KF = q^{-2} FK \,.
\end{align*}
\end{showlatex}





\section{Beware of spacings}

When typesetting a document \hologo{LaTeX} groups the appearing symbols and expressions into different groups and then adds spacing around these symbols and expressions depending on which group they belong to.
Three of these groups are \emph{operators}, \emph{relation symbols} and \emph{binary operations}.
The symbols~\texttt{=} and~\texttt{<} are for example treated as relations symbols, and the symbols~\texttt{+} and~\commandtt{cdot} as binary operations.
We can see in the following example how some space is automatically added around these symbols:
\begin{showlatex}{Standard spacing around relation symbols and binary operators}
\[
  a = b  \qquad  a < b  \qquad  a + b  \qquad  a \cdot b
\]
\end{showlatex}
To compare this to a version without spacing we can surround the symbols by a pair of curly brackets.
This circumvents \hologo{LaTeX} from taking the surround code into consideration.
This leads to the following result:
\begin{showlatex}{Disabling the standard spacing around a symbol}
\[
  a {=} b   \qquad  a {<} b  \qquad  a {+} b  \qquad  a {\cdot} b
\]
\end{showlatex}
The automatic spacing can become a problem, as the following examples illustrate:
\begin{showlatex}*{Clashing spacings around symbols}
\[
  X/\sim
  \quad
  R/\operatorname{J}(R)
  \quad
  \operatorname{id} \otimes h
\]
\end{showlatex}
This problem can be fixed by surround the respective symbols in curly brackets.
\begin{showlatex}*{Preventing a clash of spacings}
\[
  X/{\sim}
  \quad
  R/{\operatorname{J}(R)}
  \quad
  {\operatorname{id}} \otimes h
\]
\end{showlatex}
One can tell \hologo{LaTeX} how to treat a certain symbol:
\begin{showlatex}*{Specifying the role (and thus spacing) of a symbol}
\[
  a | b
  \quad
  a \mathop{|} b
  \quad
  a \mathrel{|} b
  \quad
  a \mathbin{|} b
\]
\end{showlatex}
To define a command \commandtt{divides} to express that a number $n$ divides a number $m$ we do therefore do the following:
\begin{showlatex}*{Defining and using \commandtt{divides}}
\newcommand{\divides}{\mathrel{|}}
\[
  n \divides m
\]
\end{showlatex}
For more on this topic see \cite{tex_binrel}.





\section{\texorpdfstring{\commandname{tag}}{{\tbs}tag} and \texorpdfstring{\commandname{notag}}{{\tbs}notag}}

A finer contral of tags can be achived using~\commandname{tag} and~\commandname{notag}:





\subsection{Don’t number just every formual}
\label{dont number all formulas}

Don’t indiscriminately number every formula.
Instead an equation should be numbered only if it needs to be refered to later on.



\subsection{\texorpdfstring{\commandname{tag}}{{\tbs}tag}}

With the command~\commandname{tag} a custom tag can be set.
This is useful for marking selected equations by special symbols:
\begin{showlatex}{Using~\commandname{tag} for marking a line}
Consider the equation
\begin{equation}
\label{important equation}
  2 + 2 = 5 \,.
  \tag{\ast}
\end{equation}
Note that \cref{important equation} can equivalently be expressed as~$5 = 2 + 2$.
\end{showlatex}
The argument of~\commandname{tag} is in text mode, and the resulting tag is automatically enclosed in parentheses.
These parentheses can be removed by using the starred command~\commandname{tag*} instead.

The command~\commandname{tag} should not be used for regular numbering of equations.
It should be used to tag only certain (often a single) equations in a special way.
It can also be used to express that certain transformations have been used, as the following example demonstrates:
\begin{showlatex}{Using~\commandname{tag} to explain steps}
It follows from the Chinese remainder theorem that
\begin{align*}
  \mathbb{R}[x] / ( x^3 + x^2 + x + 1 )
  &=
  \mathbb{R}[x] / ( (x^2 + 1) (x + 1) )
  \\
  &\cong
  \mathbb{R}[x] / ( x^2 + 1 ) \times \mathbb{R}[x] / ( x + 1 )
  \tag{CRT}
  \\
  &\cong
  \mathbb{C} \times \mathbb{R}
\end{align*}
\end{showlatex}



\subsection{\texorpdfstring{\commandname{notag}}{{\tbs}notag}}

According to \cref{dont number all formulas} a formula should be numbered only if it needs to be refered to.
But if this formula occurs in a multiline environment like~\envname{align*} then by switching~\envname{align} all lines will be numbered.
The prevent the numbering of the unrequired lines the command~\commandname{notag} can then be used:
\begin{showlatex}*{Using~\commandname{notag} to prevent selected line numbers}
\begin{align}
  a
  &= b \notag \\
  &= c \\
  &= d \notag \\
  &= e
\end{align}
\end{showlatex}





\section{Multiline set descriptions}

Multiline set descriptions of the form
\[
  \left\{
    (e_1, \dotsc, e_n)
  \,\middle|\,
    \begin{tabular}{@{}c@{}}
      $e_1, \dotsc, e_n \in R$ \\
      is a complete set of \\
      pairwise orthogonal \\
      idempotents
    \end{tabular}
  \right\}
\]
can be typeset by using a \envname{tabular} environment in the second argument:
\begin{showlatex}*{Multiline set descriptions with tabular}
\[
  \left\{
    x \in X
  \,\middle|\,
    \begin{tabular}{@{}c@{}}
      $x$ satisfies \\
      certain conditions
    \end{tabular}
  \right\}
\]
\end{showlatex}
Note that the entries of the environment~\envname{tabular} are automatically in text mode.
The argument~\inlinecode{\@\{\}} ensure that the environment~\envname{tabular} does not insert additional spacing to its left and right.





\section{Don’t force fancy fractions}

When fractions are placed inline, as an exponent or in an index then they should be of the form $a/b$.
The notation
\[
  \frac{a}{b}
\]
is reserved for display style.
So don’t do the following:
\begin{showlatex}{Full fractions in exponent, index and inline}
Consider $e^{\frac{1}{x}}$ and $x_{\frac{1}{n}}$ and $\frac{2}{3}$.
\end{showlatex}
Do the following instead:
\begin{showlatex}{Flat fractions in exponent, index and inline}
Consider $e^{1/x}$ and $x_{1/n}$ and $2/3$.
\end{showlatex}

Don’t use funky fractions like $\faktor{a}{b}$, they’ll burn down your house and make you go blind.

% TODO: Finding out the proper term for flat fractions. See Chicago Manual?




